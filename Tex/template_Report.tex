\documentclass[]{report}
\usepackage{polski}
\usepackage[utf8]{inputenc}


% Title Page
\title{Użycie języka Swift i elementów paradygmatu reaktywnego na przykładzie aplikacji mobilnej do polecania filmów i seriali}
\author{Bartosz Woliński}

\begin{document}
\maketitle
\tableofcontents
\chapter{WSTĘP}
\chapter{CEL I ZAKRES PRACY}
\chapter{ŹRÓDŁA I DEFINICJE}
\section{Historia systemu iOS}
\section{Historia języka Swift}
\section{Paragydmat funkcyjny}
\section{Paragydmat reaktywny}
\subsection{Idea programowania reaktywnego}
\subsection{Podejście reaktywne w praktyce}
\subsection{Operatory reaktywne}
\subsection{Współbieżność}
\subsection{Wady i zalety podejścia reaktywnego} 
\subsection{Implementacja na platformie iOS}
\chapter{PRACA WŁASNA}
\section{Czynności przygotowawcze}
\subsection{Instalacja narzędzi}
\subsection{Wstępna konfiguracja projektu}
\subsection{Stworzenie repozytorium}
\section{Budowa aplikacji właściwej}
\subsection{Wymagania funkcjonalne}
\subsection{Wymagania niefunkcjonalne}
\subsection{Diagram przypadków użycia aplikacji}
\subsection{Zależności w bazie danych}
\subsection{Diagram klas}
\subsection{Opis wybranych klas}
\subsection{Opis użytych bibliotek i frameworków}
\subsection{Implementacja aplikacji mobilnej - zastosowana architektura}
\subsection{Prezentacja aplikacji}
\chapter{PODSUMOWANIE}

\end{document}          
